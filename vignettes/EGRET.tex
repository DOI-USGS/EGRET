%\VignetteIndexEntry{Introduction to the EGRET package}
%\VignetteDepends{}
%\VignetteSuggests{}
%\VignetteImports{}
%\VignettePackage{}

\documentclass[a4paper,11pt]{article}

\usepackage{amsmath}
\usepackage{times}
\usepackage{hyperref}
\usepackage[numbers, round]{natbib}
\usepackage[american]{babel}
\usepackage{authblk}
\usepackage{subcaption}
\usepackage{placeins}
\usepackage{footnote}
\usepackage{tabularx}
\renewcommand\Affilfont{\itshape\small}
\usepackage{Sweave}
\renewcommand{\topfraction}{0.85}
\renewcommand{\textfraction}{0.1}
\usepackage{graphicx}

\textwidth=6.2in
\textheight=8.5in
\parskip=.3cm
\oddsidemargin=.1in
\evensidemargin=.1in
\headheight=-.3in

%------------------------------------------------------------
% newcommand
%------------------------------------------------------------
\newcommand{\scscst}{\scriptscriptstyle}
\newcommand{\scst}{\scriptstyle}
\newcommand{\Robject}[1]{{\texttt{#1}}}
\newcommand{\Rfunction}[1]{{\texttt{#1}}}
\newcommand{\Rclass}[1]{\textit{#1}}
\newcommand{\Rpackage}[1]{\textit{#1}}
\newcommand{\Rexpression}[1]{\texttt{#1}}
\newcommand{\Rmethod}[1]{{\texttt{#1}}}
\newcommand{\Rfunarg}[1]{{\texttt{#1}}}

\begin{document}
\Sconcordance{concordance:EGRET.tex:EGRET.Rnw:%
1 75 1 1 8 14 1 1 2 1 0 1 1 3 0 1 2 6 1 1 10 24 0 1 2 %
2 1 1 10 18 0 1 2 8 1 1 11 26 0 1 2 3 1 1 10 15 0 1 2 %
33 1 1 9 23 0 1 2 12 1 1 2 13 0 1 2 2 1 1 2 19 0 1 2 %
12 1 1 3 2 0 2 1 1 2 2 1 3 0 1 2 2 1 1 9 16 0 1 2 1 1 %
1 2 4 0 1 2 1 1 1 2 4 0 1 2 2 1 1 9 20 0 1 2 2 1 1 2 %
4 0 1 2 15 1 1 4 6 0 1 2 3 1 1 -5 1 9 12 1 1 2 1 0 1 %
1 3 0 1 2 2 1 1 2 1 0 1 1 3 0 1 2 4 1 1 5 1 -2 1 6 7 %
1 1 5 1 -2 1 6 8 1 1 9 4 1 1 -6 1 10 7 1 1 5 1 -2 1 6 %
15 1 1 2 4 0 1 2 3 1 1 -5 1 9 10 1 1 2 1 0 1 1 3 0 1 %
2 3 1 1 -5 1 9 9 1 1 2 4 0 1 2 3 1 1 -5 1 9 17 1 1 2 %
4 0 1 2 31 1 1 2 23 0 1 2 12 1 1 2 1 0 4 1 5 0 4 1 12 %
0 1 2 36 1 1 2 1 0 7 1 3 0 1 2 1 4 4 1 1 -6 1 10 7 1 %
1 4 1 -2 1 6 8 1 1 4 4 1 1 -6 1 10 7 1 1 4 1 -2 1 6 7 %
1 1 4 4 1 1 -6 1 10 7 1 1 4 1 -2 1 6 8 1 1 4 4 1 1 -6 %
1 10 9 1 1 2 4 0 1 2 3 1 1 -5 1 9 12 1 1 2 1 0 1 1 3 %
0 1 2 24 1 1 2 4 0 1 2 1 1 1 2 14 0 1 2 6 1 1 2 1 0 1 %
2 1 0 1 2 4 0 1 2 3 1 1 2 1 0 1 1 3 0 1 2 7 1 1 2 1 0 %
2 1 3 0 1 2 1 1 1 2 1 0 1 1 3 0 1 2 20 1}


%------------------------------------------------------------
\title{Introduction to the EGRET package}
%------------------------------------------------------------
\author[1]{Robert Hirsch}
\author[1]{Laura De Cicco}
\affil[1]{United States Geological Survey}



\maketitle
\tableofcontents

%------------------------------------------------------------
\section{Introduction to Exploration and Graphics for RivEr Trends (EGRET)}
%------------------------------------------------------------ 

For information on getting started in R, downloading and installing the package, see Appendix 1: (\ref{sec:appendix1}).

Exploration and Graphics for RivEr Trends (EGRET): An R-package for the analysis of long-term changes in water quality and streamflow. EGRET includes statistics and graphics for streamflow history, water quality trends, and the modeling algorithm Weighted Regressions on Time, Discharge, and Season (WRTDS). 

\textbf{Please see the official EGRET manual:}
(\href{https://github.com/USGS-R/EGRET/raw/Documentation/EGRET%2Bmanual_4.doc}{link to download}) 
\textbf{for more information on the EGRET package.}


The best way to learn about the WRTDS approach and to see examples of its application to multiple large data sets is to read two journal articles.  Both are available, for free, from the journals in which they were published.

The first relates to nitrate and total phosphorus data for 9 rivers draining to Chesapeake Bay.  The URL is \cite{HirschII}: 
\url{http://onlinelibrary.wiley.com/doi/10.1111/j.1752-1688.2010.00482.x/full}

The second is an application to nitrate data for 8 monitoring sites on the Mississippi River or its major tributaries \cite{HirschIII}.  The URL is: \url{http://pubs.acs.org/doi/abs/10.1021/es201221s}

This vignette assumes that the user understands the concepts underlying WRTDS.  Thus, reading at least the first of these papers is necessary for understanding.  The method has been enhanced beyond what was published there.  The enhancement is that it now properly handles censored data by using survival regression rather than ordinary regression.  The details of that are in a manuscript currently in process by Doug Moyer and Bob Hirsch.


This vignette will walk through the major functions provided by the EGRET package. The package dataRetrieval is required for importing data in a EGRET-friendly format. The dataRetrieval package, along with installation instructions can be found at:
\\
\url{https://github.com/USGS-R/dataRetrieval}
\\
Installing dataRetrieval will provide a vignette similar to this document, with complete working examples of the main dataRetrieval functions.

This vignette is divided into four sections: EGRET Dataframes, Flow History, WRTDS Analysis, and WRTDS Results. This document assumes the reader is familiar with the dataRetrieval package. The examples will follow an analysis of nitrate on the Choptank River at Greensboro, MD. Further details can be found in the user guide that can be found on gitHub: \url{https://github.com/USGS-R/EGRET/raw/Documentation/EGRET%2Bmanual_4.doc}


%------------------------------------------------------------ 
\section{EGRET Dataframes and Units}
\label{sec:dataframes}
%------------------------------------------------------------ 
The EGRET package uses 3 default dataframes throughout the calculations, analysis, and graphing. These dataframes are Daily (\ref{sec:dataframesDaily}), Sample (\ref{sec:dataframesSample}), and INFO (\ref{sec:dataframesINFO}). EGRET uses entirely SI units to store the data, but for purposes of output, it can report results in a wide variety of units, which will be discussed in (\ref{sec:units}). To start our exploration, the packages must be installed (check the appendix for detailed instructions (\ref{sec:appendix1})), then opened:
\begin{Schunk}
\begin{Sinput}
> library(dataRetrieval)
> library(EGRET)
\end{Sinput}
\end{Schunk}

%------------------------------------------------------------ 
\subsection{Daily}
\label{sec:dataframesDaily}
%------------------------------------------------------------ 
The Daily dataframe initially is populated with columns generated by the dataRetrieval package (Table \ref{table:Daily1}).  After running the WRTDS calculations (as will be described in \ref{sec:wrtds}), additional columns are inserted (Table \ref{table:Daily2}).

\begin{table}[!ht]
\centering
\caption{Daily dataframe} 
\label{table:Daily1}
\begin{tabular}{llll}
  \hline
ColumnName & Type & Description & Units \\ 
  \hline
Date & Date & Date & date \\ 
  Q & number & Discharge in cms & cms \\ 
  Julian & number & Number of days since January 1, 1850 & days \\ 
  Month & integer & Month of the year [1-12] & months \\ 
  Day & integer & Day of the year [1-366] & days \\ 
  DecYear & number & Decimal year & years \\ 
  MonthSeq & integer & Number of months since January 1, 1850 & months \\ 
  Qualifier & string & Qualifing code & character \\ 
  i & integer & Index & days \\ 
  LogQ & number & Natural logarithm of Q & numeric \\ 
  Q7 & number & 7 day running average of Q & cms \\ 
  Q30 & number & 30 running average of Q & cms \\ 
   \hline
\end{tabular}
\end{table}

\begin{table}[!ht]
\centering
\caption{Daily dataframe, post-WRTDS} 
\label{table:Daily2}
\begin{tabular}{llll}
  \hline
ColumnName & Type & Description & Units \\ 
  \hline
yHat & number & The WRTDS estimate of the log of concentration & numeric \\ 
  SE & number & The WRTDS estimate of the standard error of yHat & numeric \\ 
  ConcDay & number & The WRTDS estimate of concentration & mg/L \\ 
  FluxDay & number & The WRTDS estimate of flux & kg/day \\ 
  FNConc & number & Flow normalized estimate of concentration & mg/L \\ 
  FNFlux & number & Flow Normalized estimate of flux & kg/day \\ 
   \hline
\end{tabular}
\end{table}

\FloatBarrier

%------------------------------------------------------------ 
\subsection{Sample}
\label{sec:dataframesSample}
%------------------------------------------------------------ 
The Sample dataframe initially is populated with columns generated by the dataRetrieval package (Table \ref{table:Sample1}). After running the WRTDS calculations (as will be described in \ref{sec:wrtds}), additional columns are inserted (Table \ref{table:Sample2}):

\begin{table}[!ht]
\begin{minipage}{\linewidth}
\centering
\caption{Sample dataframe} 
\label{table:Sample1}
\begin{tabular}{llll}
  \hline
ColumnName & Type & Description & Units \\ 
  \hline
Date & Date & Date & date \\ 
  ConcLow & number & Lower limit of concentration & mg/L \\ 
  ConcHigh & number & Upper limit of concentration & mg/L \\ 
  Uncen & integer & Uncensored data (1=true, 0=false) & integer \\ 
  ConcAve & number & Average concentration & mg/L \\ 
  Julian & number & Number of days since January 1, 1850 & days \\ 
  Month & integer & Month of the year [1-12] & months \\ 
  Day & integer & Day of the year [1-366] & days \\ 
  DecYear & number & Decimal year & years \\ 
  MonthSeq & integer & Number of months since January 1, 1850 & months \\ 
  SinDY & number & Sine of DecYear & numeric \\ 
  CosDY & number & Cosine of DecYear & numeric \\ 
  Q \footnote{Populated after calling mergeReport} & number & Discharge & cms \\ 
  LogQ \footnote{Populated after calling mergeReport} & number & Natural logarithm of flow & numeric \\ 
   \hline
\end{tabular}
\end{minipage}
\end{table}

\begin{table}[!ht]
\centering
\caption{Sample dataframe, post-WRTDS} 
\label{table:Sample2}
\begin{tabular}{llll}
  \hline
ColumnName & Type & Description & Units \\ 
  \hline
yHat & number & jack-knife estimate of the log of concentration & numeric \\ 
  SE & number & jack-knife estimate of the standard error of yHat & numeric \\ 
  ConcHat & number & jack-knife unbiased estimate of concentration & mg/L \\ 
   \hline
\end{tabular}
\end{table}

\FloatBarrier

%------------------------------------------------------------ 
\subsection{INFO}
\label{sec:dataframesINFO}
%------------------------------------------------------------ 
The INFO dataframe is used to store information about the measurements, such as station name, parameter name, drainage area, etc. There can be many additional, optional columns, but the columns in Table \ref{table:Info1} are required to initiate the EGRET analysis. After running the WRTDS calculations (as will be described in \ref{sec:wrtds}), additional columns (Table \ref{table:Info2}) are automatically inserted into the INFO dataframe (the meaning of the values will be discussed further sections):


\begin{table}[!ht]
\begin{minipage}{\linewidth}
\begin{center}
\caption{INFO dataframe}
\label{table:Info1}
\begin{tabular}{lll}
  \hline
ColumnName & Type & Description \\ 
  \hline
shortName & string & Name of site, suitable for use in graphical headings \\ 
  staAbbrev & string & Abbreviation for station name, used in saveResults \\ 
  paramShortName & string & Name of constituent, suitable for use in graphical headings \\ 
  constitAbbrev & string & Abbreviation for constituent name, used in saveResults \\ 
  drainSqKm & numeric & Drainage area in  km\verb@^@2 \\ 
  paStart \footnote{Inserted with the setPA function} & integer (1-12) & Starting month of period of analysis \\ 
  paLong \footnote{Inserted with the setPA function} & integer (1-12) & Length of period of analysis in months \\ 
   \hline
\end{tabular}
\end{center}
\end{minipage}
\end{table}

\begin{table}[!ht]
\centering
\caption{INFO dataframe, post-WRTDS} 
\label{table:Info2}
\begin{tabular}{lll}
  \hline
ColumnName & Description & Units \\ 
  \hline
bottomLogQ & Lowest discharge in prediction surfaces & numeric \\ 
  stepLogQ & Step size in discharge in prediction surfaces & numeric \\ 
  nVectorLogQ & Number of steps in discharge, prediction surfaces & numeric \\ 
  bottomYear & Starting year in prediction surfaces & numeric \\ 
  stepYear & Step size in years in prediction surfaces & numeric \\ 
  nVectorYear & Number of steps in years in prediction surfaces & numeric \\ 
  windowY & Half-window width in the time dimension & years \\ 
  windowQ & Half-window width in the log discharge dimension & numeric \\ 
  windowS & Half-window width in the seasonal dimension & years \\ 
  minNumObs & Minimum number of observations for regression & integer \\ 
  minNumUncen & Minimum number of uncensored observations & integer \\ 
   \hline
\end{tabular}
\end{table}

\FloatBarrier

%------------------------------------------------------------ 
\subsection{Units}
\label{sec:units}
%------------------------------------------------------------ 
EGRET uses entirely SI units to store the data, but for purposes of output, it can report results in a wide variety of units. The default is that concentration is measured in mg/L, discharge is cubic meters per second (cms), flux is kg/day, and drainage area is km\verb@^@2. When discharge values are imported from USGS web services (using the dataRetrieval package), they are automatically converted from cubic feet per second (cfs) to cms unless the argument convet is set to FALSE.  This can cause confusion if not careful. 

Although the data is stored in the dataframes in SI, it is possible to report the results in a variety of units. For all functions that provide output, there are two arguments that can be defined to set the output units: qUnit and FluxUnit.  qUnit and FluxUnit can be defined by a numeric code or name.  There are two functions that can be called to see the options for qUnit and FluxUnit: printqUnitCheatSheet and printFluxUnitCheatSheet.


\begin{Schunk}
\begin{Sinput}
> printqUnitCheatSheet()
\end{Sinput}
\begin{Soutput}
The following codes apply to the qUnit list:
1 =  cfs  ( Cubic Feet per Second )
2 =  cms  ( Cubic Meters per Second )
3 =  thousandCfs  ( Thousand Cubic Feet per Second )
4 =  thousandCms  ( Thousand Cubic Meters per Second )
5 =  mmDay  ( mm per day )
6 =  mmYear  ( mm per year )
\end{Soutput}
\end{Schunk}

When a function has an input argument qUnit, you can define the flow units with the index (1-6) as shown above. The choice should be based on the units that are customary for the audience, but also so that the discharge values don't have too many digits to the right or left of the decimal point.

\begin{Schunk}
\begin{Sinput}
> printFluxUnitCheatSheet()
\end{Sinput}
\begin{Soutput}
The following codes apply to the fluxUnit list:
1 =  poundsDay  ( pounds/day )
2 =  tonsDay  ( tons/day )
3 =  kgDay  ( kg/day )
4 =  thousandKgDay  ( thousands of kg/day )
5 =  tonsYear  ( tons/year )
6 =  thousandTonsYear  ( thousands of tons/year )
7 =  millionTonsYear  ( millions of tons/year )
8 =  thousandKgYear  ( thousands of kg/year )
9 =  millionKgYear  ( millions of kg/year )
10 =  billionKgYear  ( billions of kg/year )
11 =  thousandTonsDay  ( thousands of tons/day )
12 =  millionKgDay  ( millions of kg/day )
\end{Soutput}
\end{Schunk}

When a function has an input argument FluxUnit, you can define the flux units with the index (1-12) as shown above. The choice should be based on the units that are customary for the audience, but also so that the flux values don't have too many digits to the right or left of the decimal point.


%------------------------------------------------------------ 
\section{Flow History}
\label{sec:flowHistory}
%------------------------------------------------------------ 
This section describes functions included in the EGRET package that provide a variety of table and graphical outputs looking only at flow statistics based on time-series smoothing. These functions were designed for studies of long-term streamflow change and work best for daily streamflow data sets of 50 years or longer. This type of analysis might be useful for studying 

At this point it is assumed that you can load the daily discharge record into R, create the Daily dataframe, and enter the required meta-data into the INFO dataframe. If not, see the dataRetrieval vignette:

\begin{Schunk}
\begin{Sinput}
> vignette("dataRetrieval")
\end{Sinput}
\end{Schunk}

We will walk through an example from the Rio Grande gaging station in Embodo, NM.  This is the first stream gage station in the USGS, established by John Wesley Powell in 1888.


\begin{Schunk}
\begin{Sinput}
> #Rio Grande at Embudo, NM
> siteID <- "08279500"  
> startDate <- ""
> endDate <- ""
> Daily <- getDVData(siteID,"00060",startDate,endDate,interactive=FALSE)
> INFO <- getMetaData(siteID,"",interactive=FALSE)
> INFO$shortName <- "Rio Grande at Embudo, NM"
\end{Sinput}
\end{Schunk}

The first choice you need to make is what period of analysis to use (pa). What is the period of analysis?  If we want to examine our data set as a time series of water years, then the period of analysis is October through September.  If we want to examine the data set as calendar years then the period of analysis should be January through December.  We might want to examine the winter season, which we could define as December through February, then those 3 months become the period of analysis. The only constraints on the definition of a period of analysis are these: It must be defined in terms of whole months.  It must be a set of contiguous months (like March-April-May).  And it must have a length that is no less than 1 month and no more than 12 months.  It can be uniquely defined by two arguments: paLong and paStart.  paLong is the length of the period of analysis, and paStart is the first month of the period of analysis. Table \ref{table:paINFO} summarizes paLong and paStart.

\begin{table}[!ht]
\centering
\caption{Period of Analysis Information} 
\label{table:paINFO}
\begin{tabular}{lll}
  \hline
PeriodOfAnalysis & paStart & PaLong \\ 
  \hline
Calendar Year & 1 & 12 \\ 
  Water Year & 10 & 12 \\ 
  Winter & 12 & 3 \\ 
  September & 9 & 1 \\ 
   \hline
\end{tabular}
\end{table}

To set a period running from December through February:
\begin{Schunk}
\begin{Sinput}
> INFO <- setPA(paStart=12,paLong=3)
\end{Sinput}
\end{Schunk}

To set the default value (water year):
\begin{Schunk}
\begin{Sinput}
> INFO <- setPA()
\end{Sinput}
\end{Schunk}

The next step is to create the annual series of flow statistics.  These will be stored in a matrix called annualSeries that contain the statistics described in table \ref{table:istat}.

\begin{table}[!ht]
\centering
\caption{Index of Statistics Information} 
\label{table:istat}
\begin{tabular}{ll}
  \hline
istat & Name \\ 
  \hline
1 & 1-day minimum flow \\ 
  2 & 7-day minimum flow \\ 
  3 & 30-day minimum flow \\ 
  4 & median flow \\ 
  5 & mean flow \\ 
  6 & 30-day maximum flow \\ 
  7 & 7-day maximum flow \\ 
  8 & 1-day maximum flow \\ 
   \hline
\end{tabular}
\end{table}

To create the annualSeries matrix, using the function makeAnnualSeries:
\begin{Schunk}
\begin{Sinput}
> annualSeries <- makeAnnualSeries()
\end{Sinput}
\end{Schunk}

Once the annualSeries matrix is created, the plots of any of the stored statistics can be generated with the plotFlowSingle function.

%------------------------------------------------------------ 
\subsection{Plotting Options}
\label{sec:plotOptions}
%------------------------------------------------------------ 
This section will give examples of the available plots appropriate for studying flow history once the annualSeries has been created. The plots here will use the default variable input options.  For any function, you can get a complete list of input variables (as described in the previous section) in a help file by typing a ? before the function name in the R console. See Appendix \ref{sec:flowHistoryVariables} for information on the available input variables for these plotting functions. Also, the complete EGRET manual has more detailed information for each plot type (\href{https://github.com/USGS-R/EGRET/raw/Documentation/EGRET%2Bmanual_4.doc}{link to download}).

\begin{Schunk}
\begin{Sinput}
> Daily <- getDVData(siteID,"00060",startDate,endDate,interactive=FALSE)
> INFO <- getMetaData(siteID,"",interactive=FALSE)
> INFO$shortName <- "Rio Grande at Embudo, NM"
> INFO <- setPA()
> annualSeries <- makeAnnualSeries()
> plotFlowSingle(istat=7,qUnit="thousandCfs")
> plotSDLogQ()
> plotQTimeDaily(1990,2010,qLower=1,qUnit=3)
> plotFour(qUnit=3)
> plotFourStats(qUnit=3)
\end{Sinput}
\end{Schunk}

