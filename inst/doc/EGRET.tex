%\VignetteIndexEntry{Introduction to the dataRetrieval package}
%\VignetteDepends{}
%\VignetteSuggests{}
%\VignetteImports{}
%\VignettePackage{}

\documentclass[a4paper,11pt]{article}

\usepackage{amsmath}
\usepackage{times}
\usepackage{hyperref}
\usepackage[numbers, round]{natbib}
\usepackage[american]{babel}
\usepackage{authblk}
\renewcommand\Affilfont{\itshape\small}
\usepackage{Sweave}
\renewcommand{\topfraction}{0.85}
\renewcommand{\textfraction}{0.1}
\usepackage{graphicx}


\textwidth=6.2in
\textheight=8.5in
\parskip=.3cm
\oddsidemargin=.1in
\evensidemargin=.1in
\headheight=-.3in

%------------------------------------------------------------
% newcommand
%------------------------------------------------------------
\newcommand{\scscst}{\scriptscriptstyle}
\newcommand{\scst}{\scriptstyle}
\newcommand{\Robject}[1]{{\texttt{#1}}}
\newcommand{\Rfunction}[1]{{\texttt{#1}}}
\newcommand{\Rclass}[1]{\textit{#1}}
\newcommand{\Rpackage}[1]{\textit{#1}}
\newcommand{\Rexpression}[1]{\texttt{#1}}
\newcommand{\Rmethod}[1]{{\texttt{#1}}}
\newcommand{\Rfunarg}[1]{{\texttt{#1}}}

\begin{document}
\Sconcordance{concordance:EGRET.tex:EGRET.Rnw:%
1 82 1 1 3 5 0 1 2 2 1 1 4 6 0 1 2 2 1 1 4 6 0 1 2 3 1 1 2 1 0 1 1 3 0 %
1 2 6 1 1 2 1 0 2 1 3 0 1 2 1 1 1 2 4 0 1 2 20 1}


%------------------------------------------------------------
\title{Introduction to the EGRET package}
%------------------------------------------------------------
\author[1]{Laura De Cicco}
\author[1]{Robert Hirsch}
\affil[1]{United States Geological Survey}



\maketitle
\tableofcontents

%------------------------------------------------------------
\section{Introduction to EGRET}
%------------------------------------------------------------ 

For information on getting started in R, downloading and installing the package, see Appendix 1: Getting Started.




\newpage
%------------------------------------------------------------ 
\section{Appendix 1: Getting Started}
%------------------------------------------------------------ 
This section describes the options for downloading and installing the dataRetrieval package.

%------------------------------------------------------------
\subsection{New to R?}
%------------------------------------------------------------ 
If you are new to R, you will need to first install the latest version of R, which can be found here: \url{http://www.r-project.org/}.

There are many options for running and editing R code, one nice enviornment to learn R is RStudio. RStudio can be downloaded here: \url{http://rstudio.org/}. Once R and RStudio are installed, the dataRetrieval package needs to be installed as described in the next section.

%------------------------------------------------------------
\subsection{R User: Installing dataRetrieval from downloaded binary}
%------------------------------------------------------------ 
The latest dataRetrieval package build is available for download at \url{https://github.com/USGS-R/dataRetrieval/raw/packageBuilds/EGRET_1.2.1.tar.gz}.  If the package's tar.gz file is saved in R's working directory, then the following command will fully install the package:

\begin{Schunk}
\begin{Sinput}
> install.packages("EGRET_1.2.3.tar.gz", 
+                  repos=NULL, type="source")
\end{Sinput}
\end{Schunk}

If the downloaded file is stored in an alternative location, include the path in the install command.  A Windows example looks like this (notice the direction of the slashes, they are in the opposite direction that Windows normally creates paths):

\begin{Schunk}
\begin{Sinput}
> install.packages(
+   "C:/RPackages/Statistics/EGRET_1.2.3.tar.gz", 
+   repos=NULL, type="source")
\end{Sinput}
\end{Schunk}

A Mac example looks like this:

\begin{Schunk}
\begin{Sinput}
> install.packages(
+   "/Users/userA/RPackages/Statistic/dataRetrieval_1.2.1.tar.gz", 
+   repos=NULL, type="source")
\end{Sinput}
\end{Schunk}

It is a good idea to re-start the R enviornment after installing the package, especially if installing an updated version (that is, restart RStudio). Some users have found it necessary to delete the previous version's package folder before installing newer version of dataRetrieval. If you are experiencing issues after updating a package, trying deleting the package folder - the default location for Windows is something like this: C:/Users/userA/Documents/R/win-library/2.15/dataRetrieval, and the default for a Mac: /Users/userA/Library/R/2.15/library/dataRetrieval. Then, re-install the package using the directions above. Moving to CRAN should solve this problem.

After installing the package, you need to open the library each time you re-start R.  This is done with the simple command:
\begin{Schunk}
\begin{Sinput}
> library(dataRetrieval)
> library(EGRET)
\end{Sinput}
\end{Schunk}
Using RStudio, you could alternatively click on the checkbox for dataRetrieval and EGRET in the Packages window.

%------------------------------------------------------------
\subsection{R Developers: Installing dataRetrieval from gitHub}
%------------------------------------------------------------
Alternatively, R-developers can install the latest version of dataRetrieval directly from gitHub using the devtools package.  devtools is available on CRAN.  Simpley type the following commands into R to install the latest version of dataRetrieval available on gitHub.  Rtools (for Windows) and appropriate \LaTeX\ tools are required.

\begin{Schunk}
\begin{Sinput}
> library(devtools)
> install_github("dataRetrieval", "USGS-R")
> install_github("EGRET", "USGS-R")
\end{Sinput}
\end{Schunk}
To then open the library, simply type:

\begin{Schunk}
\begin{Sinput}
> library(dataRetrieval)
\end{Sinput}
\end{Schunk}


%------------------------------------------------------------
% BIBLIO
%------------------------------------------------------------
\begin{thebibliography}{10}

\bibitem{HirschI}
Helsel, D.R. and R. M. Hirsch, 2002. Statistical Methods in Water Resources Techniques of Water Resources Investigations, Book 4, chapter A3. U.S. Geological Survey. 522 pages. \url{http://pubs.usgs.gov/twri/twri4a3/}

\bibitem{HirschII}
Hirsch, R. M., Moyer, D. L. and Archfield, S. A. (2010), Weighted Regressions on Time, Discharge, and Season (WRTDS), with an Application to Chesapeake Bay River Inputs. JAWRA Journal of the American Water Resources Association, 46: 857-880. doi: 10.1111/j.1752-1688.2010.00482.x \url{http://onlinelibrary.wiley.com/doi/10.1111/j.1752-1688.2010.00482.x/full}

\bibitem{HirschIII}
Sprague, L. A., Hirsch, R. M., and Aulenbach, B. T. (2011), Nitrate in the Mississippi River and Its Tributaries, 1980 to 2008: Are We Making Progress? Environmental Science \& Technology, 45 (17): 7209-7216. doi: 10.1021/es201221s \url{http://pubs.acs.org/doi/abs/10.1021/es201221s}

\end{thebibliography}

\end{document}

\end{document}
